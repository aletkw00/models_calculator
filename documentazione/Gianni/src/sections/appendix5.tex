%! Author = Gianni
%! Date = 17/10/2023


\chapter{Configurazioni Home Assistant}
\label{ch:hass-appendix}


\section{Definizioni di sensori MQTT}
\label{sec:hass-appendix-sensori}
Qui sono riportate le definizioni di sensori MQTT 
usate nel file configuration.yaml.

\subsection*{Sensore con valore semplice}
\label{sec:hass-appendix-sensori-normali}
La configurazione di un sensore con valore numerico intero. 
\begin{lstlisting}[language=bash]
    - name: Serra reale Temperatura Aria
      state_topic: sensori
      unique_id: sensor.reali.serraAirTemperature
      unit_of_measurement: "°C"
      value_template: >
        
          {{ value_json.Temp__C }}
        
          {{ sensor.reali.serraAirTemperature }}
        
\end{lstlisting}


\subsection*{Sensore con valore arrotondato}
\label{sec:hass-appendix-sensori-float}
La configurazione di un sensore con valore numerico con la virgola 
che viene arrotondato alla seconda cifra decimale.
\begin{lstlisting}[language=bash]
    - name: Serra stima RL Temperatura 1
      state_topic: sensori/modelliRL
      unique_id: sensor.PredRL.serraTemperatura_1
      unit_of_measurement: "°C"
      value_template: >
        
          {{ value_json.RL_1 | float | round(2) }}
        
          {{ sensor.PredRL.serraTemperatura_1 }}
        
\end{lstlisting}


\section{Definizioni di cards}
\label{sec:hass-appendix-plancia}
Di seguito viene riportato il codice usato per la creazione
delle card nell'interfaccia grafica.

\subsection*{Card per le entità}
\label{sec:hass-appendix-plancia-entità}
Permette di mostrare il nome e il valore del sensore.
\begin{lstlisting}[language=bash]
    - type: entities
      title: Temperatura
      entities:
        - entity: sensor.serra_reale_temperatura_aria
          name: Temperatura Aria
        - entity: sensor.serra_reale_massima_temperatura_aria
          name: Temperatura Aria massima
        - entity: sensor.serra_reale_minima_temperatura_aria
          name: Temperatura Aria minima    
\end{lstlisting}

\subsection*{Card per immagini-elementi}
\label{sec:hass-appendix-plancia-pic-ele}
Permette di mostrare un'immagine con sopra dei badge 
che contengono i valori dei sensori.
\begin{lstlisting}[language=bash]
    - type: picture-elements
      title: Stime RL
      image: http://mydomain.it:8123/local/Mappa_new.png
      elements:
        - type: state-badge
          entity: sensor.serra_stima_rl_temperatura_1
          style:
            top: 28%
            left: 37%
            scale: 95%
            color: rgba(0,0,0,0)
        - type: state-badge
          entity: sensor.serra_stima_rl_temperatura_3
          style:
            top: 50%
            left: 37%
            scale: 95%
            color: rgba(0,0,0,0)
        - type: state-badge
          entity: sensor.serra_stima_rl_temperatura_5
          style:
            top: 74%
            left: 37%
            scale: 95%
            color: rgba(0,0,0,0)
        - type: state-badge
          entity: sensor.serra_stima_rl_temperatura_7
          style:
            top: 95%
            left: 37%
            scale: 95%
            color: rgba(0,0,0,0)
        - type: state-badge
          entity: sensor.serra_stima_rl_temperatura_2
          style:
            top: 28%
            left: 62%
            scale: 95%
            color: rgba(0,0,0,0)
        - type: state-badge
          entity: sensor.serra_stima_rl_temperatura_4
          style:
            top: 50%
            left: 62%
            scale: 95%
            color: rgba(0,0,0,0)
        - type: state-badge
          entity: sensor.serra_stima_rl_temperatura_6
          style:
            top: 74%
            left: 62%
            scale: 95%
            color: rgba(0,0,0,0)
        - type: state-badge
          entity: sensor.serra_stima_rl_temperatura_8
          style:
            top: 95%
            left: 62%
            scale: 95%
            color: rgba(0,0,0,0)
\end{lstlisting}

\subsection*{Card per iframe}
\label{sec:hass-appendix-plancia-iframe}
Permette di mostrare una pagina web all'interno di un elemento.
È stato usato per inserire i grafici di Grafana.
\begin{lstlisting}[language=bash]
    - type: iframe
      url: >- url_to_web_page_to_embed
      aspect_ratio: 40% 
\end{lstlisting}