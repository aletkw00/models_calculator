%! Author = Gianni
%! Date = 17/10/2023


\chapter{Configurazioni broker MQTT}
\label{ch:broker-configurazioni-broker-mqtt}

\section{/mosquitto.conf}
\label{sec:/mosquitto.conf}
Questa configurazione è quella che mosquitto dovrebbe creare al momento dell'installazione.
Se non è presente potrebbe non funzionare.\newline
Il seguente file deve essere inserito in \textbf{/etc/mosquitto}.
\begin{lstlisting}[caption={mosquitto.conf},language=text]
# Place your local configuration in /etc/mosquitto/conf.d/
#
# A full description of the configuration file is at
# /usr/share/doc/mosquitto/examples/mosquitto.conf.example

pid_file /var/run/mosquitto.pid

persistence true
persistence_location /var/lib/mosquitto/

log_dest file /var/log/mosquitto/mosquitto.log

include_dir /etc/mosquitto/conf.d
\end{lstlisting}

\section{/conf.d/localhost-TLS.config}
\label{sec:/conf.d/localhost-tls.config}
Questa configurazione è quella creata ad hoc.
Ha 2 listener separati, uno per localhost con accesso libero, uno per tutti con accesso tramite
credenziali e connessione tramite SSL/TLS.\newline
Il seguente file deve essere inserito in \textbf{/etc/mosquitto/conf.d}.
\begin{lstlisting}[caption={mosquitto.conf},language=text]
# abilita differenti configurazioni per listener
per_listener_settings true
#
# per accesso solo locale e libero
listener 1883 localhost
allow_anonymous true
#
# per accesso con credenziali e certificato SSL/TLS
listener 8883
allow_anonymous false
password_file /etc/mosquitto/passwordfile
certfile /etc/mosquitto/certs/fullchain.pem
keyfile /etc/mosquitto/certs/privkey.pem
\end{lstlisting}

\section{Script rinnovo certificati}
\label{sec:script-rinnovo-certificati}
Questo script deve essere messo nella cartella /etc/letsencrypt/renewal-hooks/deploy/ e reso eseguibile.
Copia i certificati del dominio inserito nella cartella /etc/mosquitto/certs/ in modo che Mosquitto possa accedervi.
Quando copia, invia a Mosquitto un segnale per ricaricare i certificati.
\begin{lstlisting}[caption={mosquitto-copy.sh},language=text]
#!/bin/sh

# This is an example deploy renewal hook for certbot that copies newly updated
# certificates to the Mosquitto certificates directory and sets the ownership
# and permissions so only the mosquitto user can access them, then signals
# Mosquitto to reload certificates.

# RENEWED_DOMAINS will match the domains being renewed for that certificate, so
# may be just "example.com", or multiple domains "www.example.com example.com"
# depending on your certificate.

# Place this script in /etc/letsencrypt/renewal-hooks/deploy/ and make it
# executable after editing it to your needs.

# Set which domain this script will be run for
MY_DOMAIN=example.com
# Set the directory that the certificates will be copied to.
CERTIFICATE_DIR=/etc/mosquitto/certs

for D in ${RENEWED_DOMAINS}; do
	if [ "${D}" = "${MY_DOMAIN}" ]; then
		# Copy new certificate to Mosquitto directory
		cp ${RENEWED_LINEAGE}/fullchain.pem ${CERTIFICATE_DIR}/server.pem
		cp ${RENEWED_LINEAGE}/privkey.pem ${CERTIFICATE_DIR}/server.key

		# Set ownership to Mosquitto
		chown mosquitto: ${CERTIFICATE_DIR}/server.pem ${CERTIFICATE_DIR}/server.key

		# Ensure permissions are restrictive
		chmod 0600 ${CERTIFICATE_DIR}/server.pem ${CERTIFICATE_DIR}/server.key

		# Tell Mosquitto to reload certificates and configuration
		pkill -HUP -x mosquitto
	fi
done
\end{lstlisting}