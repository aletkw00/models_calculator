%! Author = Gianni
%! Date = 17/10/2023


\chapter{Skeleton client MQTT}
\label{ch:client-appendix}

\section*{config.py}
\label{sec:client-appendix-subpub-conf}
File di configurazione per lo script principale.
\begin{lstlisting}[language=bash]
    """
    Configurazione 
    """
    
    # Config. per connessione
    HOST = 'host'
    USER = 'user'
    PASSWORD = 'pass'
    CERT_TLS = 'ISRG_Root_X1.pem'
    
    # Nomi dei topic ricezione
    TOPIC_RECEIVE_1 = "topic1"
    TOPIC_RECEIVE_2 = "topic2"
    
    # Nome topic invio
    TOPIC_SEND_1 = "topic3"
    TOPIC_SEND_2 = "topic4"
\end{lstlisting}

\section*{subscriber\_e\_publisher.py}
\label{sec:client-appendix-subpub-main}
File principale in cui ci si iscrive a dei topic.
Alla ricezione di messaggi si elaborano e poi si pubblica il risultato.
\begin{lstlisting}[language=bash]
    import paho.mqtt.client as mqtt
    import json
    
    import config
      
    #------------------------
    # CLASSE PER I PARAMETRI
    #
    class configurazione_localhost:
        def __init__(self, topic_ricezione_1, topic_ricezione_2, topic_invio_1, topic_invio_2):
            self.dizionario = {"host": "localhost",
                               "porta": 1883,
                               "login": False,
                               "username": "",
                               "password": "",
                               "topic-ricezione-1": topic_ricezione_1,
                               "topic-ricezione-2": topic_ricezione_2,
                               "topic-invio-1": topic_invio_1,
                               "topic-invio-2": topic_invio_2,
                               "cert_tls": ''}
        
        def aggiorna_ad_esterna_login_tls(self, host, username, password, nome_cert):
            self.dizionario.update({"host": host})
            self.dizionario.update({"porta": 8883})
            self.dizionario.update({"login": True})
            self.dizionario.update({"username": username})
            self.dizionario.update({"password": password})
            self.dizionario.update({"cert_tls": nome_cert})
    
        def get_host(self): 
            return self.dizionario.get('host')
        
        def get_porta(self):
            return int(self.dizionario.get('porta'))
    
        def get_login(self):
            return bool(self.dizionario.get('login'))
    
        def get_username(self):
            return self.dizionario.get('username')
    
        def get_password(self):
            return self.dizionario.get('password')
    
        def get_topic_ric_1(self):
            return self.dizionario.get('topic-ricezione-1')
            
        def get_topic_ric_2(self):
            return self.dizionario.get('topic-ricezione-2')
        
        def get_topic_inv_1(self):
            return self.dizionario.get('topic-invio-1')
        
        def get_topic_inv_2(self):
            return self.dizionario.get('topic-invio-2')
        
        def get_cert_tls(self):
            return self.dizionario.get('cert_tls')
    
    #-------------------------------------
    # FUNZIONE PER LA ELABORAZIONE ALLA RECEZIONE DI UN MESSAGGIO
    # L'output deve essere un json/dizionario
    #
    def elaborazione(msg, topic, par):
        global variabile_globale_1
        global variabile_globale_2
        
        # DEBUG - Stampa tutto il contenuto del messaggio
        # print(msg)
        
        # Converte il messaggio in un json (dizionario)
        msg_json = json.loads(msg)
    
    
        # Esegue il dump del json
        nuovo_msg = json.dumps(msg_json)
        
        # TOGLIERE IL COMMENTO PER DEBUG
        # print(nuovo_msg) 
        return nuovo_msg
    
    #----------------------------------
    # CLIENT MQTT
    # definizione di cosa deve fare alla connessione, iscrizione, ricezione, invio, log
    #
    def sub_pub_mqtt(par: configurazione_localhost):
    
        def on_connect(client, userdata, flags, rc):
            # TOGLIERE IL COMMENTO PER DEBUG
            # print(client, userdata, flags, rc)
            pass
    
        def on_subscribe(client, userdata, mid, granted_qos):
            # TOGLIERE IL COMMENTO PER DEBUG
            # print("message topic=", mid)
            # print("message qos=", granted_qos)
            pass
    
        def on_message(client, userdata, msg):
            json_string = elaborazione(msg.payload.decode(), msg.topic, par)
    
            if json_string:  
                client_pub_sub.publish(
                    topic=par.get_topic_inv_1(),
                    payload=json_string,
                    retain=True
                )        
    
        def on_log(client, userdata, level, buf):
            print("log: ", buf)
    
        # crea il client per la connessione
        client_pub_sub = mqtt.Client(
            client_id=None,
            clean_session=True,
            userdata=None,
            protocol=mqtt.MQTTv311,
            transport='tcp'
        )
    
        # callback
        client_pub_sub.on_connect = on_connect  # si puo' commentare se non interessa
        client_pub_sub.on_subscribe = on_subscribe  # si puo' commentare se non interessa
        client_pub_sub.on_message = on_message
        # --------------------- 
        # PER DEBUG
        # da commentare per non vedere i log
        #client_pub_sub.on_log = on_log
    
        # controlla se c'e' il login
        if par.get_login():
            client_pub_sub.username_pw_set(par.get_username(), par.get_password())
    
        # controlla la porta di invio e setta il certificato
        if par.get_porta() != 1883:
            client_pub_sub.tls_set(ca_certs=par.get_cert_tls(), tls_version=2)
    
        # esegue la connessione al broker
        client_pub_sub.connect(
            host=par.get_host(),
            port=par.get_porta(),
            keepalive=60
        )
    
        # Iscrizione ai topic di ricezione
        client_pub_sub.subscribe(topic=par.get_topic_ric_1())
        # CI SI PUO' ISCRIVERE A PIU' TOPIC
        # client_pub_sub.subscribe(topic=par.get_topic_ric_2())
    
        # attesa dei messaggi all'infinito, serve per eseguire i callback
        client_pub_sub.loop_forever(timeout=1.0, max_packets=1, retry_first_connection=False)

    #-------------------------------------------
    # FUNZIONE MAIN
    if __name__ == "__main__":
    
        # definizione di variabili globali
        global variabile_globale_1
        global variabile_globale_2
        
        variabile_globale_1  = False
        variabile_globale_2 = False
        
        # Connessione al broker e avvio del client mqtt
        parametri = configurazione_localhost(config.TOPIC_RECEIVE_1, config.TOPIC_RECEIVE_2, config.TOPIC_SEND_1, config.TOPIC_SEND_2)
        #
        # togliere il commento per configurazione esterna alla macchina
        #parametri.aggiorna_ad_esterna_login_tls(config.HOST, config.USER, config.PASSWORD, config.CERT_TLS)
    
        sub_pub_mqtt(parametri)
\end{lstlisting}