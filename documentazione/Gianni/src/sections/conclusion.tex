%! Author = Gianni
%! Date = 08/02/2024

\chapter*{Conclusioni}
\addcontentsline{toc}{chapter}{\protect\numberline{}Conclusioni}

Lo svolgimento del progetto mi ha permesso di applicare varie conoscenze 
apprese durante gli anni universitari.
Da un lato ho potuto sperimentare la programmazione di complessi script Python
e dall'altro ho potuto agire come sistemista per far collaborare tutti i vari componenti.\newline
\newline
Alla base del progetto c'è il broker MQTT che fa da ponte per i dati tra le varie parti.
Tutto parte dall'emulazione di sensori reali che viene fatta da alcuni client MQTT che pubblicano dati.
Tramite l'interfaccia web vengono generati i modelli di elaborazione dei dati dei sensori.
Questi modelli vengono usati da altri client MQTT per generare nuovi dati
a partire dai sensori emulati.
Infine con Home Assistant è possibile vedere in un'interfaccia grafica tutti i dati 
che vengono pubblicati sul broker.\newline
\newline
È possibile scaricare il codice sviluppato dalla repository GitHub
al \href{https://github.com/aletkw00/models_calculator/tree/masterG-python36}{seguente link}.



\chapter*{Ringraziamenti}
\addcontentsline{toc}{chapter}{\protect\numberline{}Ringraziamenti}
Giunto alla conclusione di questa tesi voglio ringraziare una persona molto speciale,
la dottoressa Cristina Calonego, l'amore della mia vita.
Lei mi ha supportato in tutte le varie fasi del mio percorso universitario
motivandomi a studiare e a portare a termine il percorso.
Inoltre mi ha aiutato a revisionare questa tesi per renderla comprensibile a tutti
e per controllare che non ci fossero porcherie grammaticali.
Ringrazio la mia famiglia che mi ha supportato economicamente e moralmente
in questi anni nonostante tutte le mie insicurezze.
Infine voglio ringraziare il Prof. Davide Quaglia e il Dott. Elia Brentarolli per 
avermi dato la possibilità di partecipare a questo progetto 
che è stato presentato alla Fieragricola 2024 di Verona.

