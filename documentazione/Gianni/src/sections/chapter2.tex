%! Author = Gianni
%! Date = 17/10/2023

\chapter{Broker MQTT Mosquitto}
\label{ch:broker-mqtt}

\section{Implementazione nel progetto}
\label{sec:broker-introduzione}
Alla base del progetto c'è il broker che permette la comunicazione tra le varie componenti.
Si è scelto di usare come broker Eclipse Mosquitto perché è open source e implementa il protocollo MQTT fino alla versione 5.0.
Di seguito viene spiegato come installare Mosquitto da terminale e come configurarlo per l'accesso locale e da remoto.
La guida è stata pensata per sistemi Linux.

\section{Installazione}
\label{sec:broker-installazione}

\subsection{Comandi per l'installazione}
L'installazione si può eseguire in più modi, come spiegato nel sito ufficiale \cite{mosquitto-install}:

\begin{itemize}
    \item \textbf{Metodo 1} - tramite la repository ufficiale (consigliato).\\
    In questo modo viene installata l'ultima versione disponibile delle release.
    \begin{lstlisting}[language=bash]
        sudo apt-add-repository ppa:mosquitto-dev/mosquitto-ppa
        sudo apt-get update
        sudo apt-get install mosquitto
    \end{lstlisting}
    \item \textbf{Metodo 2} - tramite la repository di linux.\\
    In questo modo viene installata l'ultima versione disponibile nel package manager.
    \begin{lstlisting}[language=bash]
        sudo apt-get install mosquitto
    \end{lstlisting}
\end{itemize}

\subsection{Come controllare l'esecuzione del servizio}
Il broker MQTT viene eseguito all'avvio della macchina tramite un servizio che viene aggiunto nell'installazione.
Il servizio si trova in:
\begin{lstlisting}[language=textnonum]
    /etc/systemd/system/mosquitto.service
\end{lstlisting}
È possibile controllare il broker con i consueti comandi per la gestione di un servizio elencati di seguito.
È necessario usare sudo, quindi si devono avere i diritti di root.
\begin{itemize}
    \item lo stato del servizio:
    \newline Solo questa opzione ha la possibilità di essere eseguita senza il comando sudo.
    \begin{lstlisting}[language=bash]
        sudo service mosquitto status
    \end{lstlisting}
    \item fermare il servizio:
    \begin{lstlisting}[language=bash]
        sudo service mosquitto stop
    \end{lstlisting}
    \item avviare il servizio:
    \begin{lstlisting}[language=bash]
        sudo service mosquitto start
    \end{lstlisting}
    \item riavviare il servizio:
    \begin{lstlisting}[language=bash]
        sudo service mosquitto restart
    \end{lstlisting}
\end{itemize}
Per vedere il log dettagliato di Mosquitto, si deve guardare nel file:
\begin{lstlisting}[language=textnonum]
    /var/log/mosquitto/mosquitto.log
\end{lstlisting}


\subsection{Problemi nell'installazione}
Si potrebbero presentare principalmente 2 tipi di problemi:
\begin{itemize}
    \item Mancata creazione della configurazione.\newline
    Vedere nella sezione~\ref{subsec:broker-esempi-configurazione} in cui si possono trovare esempi di configurazione.
    \item Nel caso di una precedente installazione:
    \begin{itemize}
        \item Si potrebbe avere un errore di database.\newline
        In questo caso si deve cancellare il database che si trova in:\\
        \textbf{var/lib/mosquitto/mosquitto.db} \newline
        Alla successiva esecuzione di Mosquitto questo database viene ricreato.
        \item Si potrebbero avere file di configurazione sbagliati o corrotti.\newline
        In questo caso si consiglia di creare un backup completo della cartella \textbf{/etc/mosquitto}.
        Eliminare tutti i file nella sotto cartella conf.d, rieseguire Mosquitto e vedere se funziona.
        Se il problema persiste eliminare anche il file con estensione .conf nella cartella mosquitto.
    \end{itemize}
\end{itemize}


\section{Configurazione}
\label{sub:broker-configurazione-del-broker}
Viene riportata la struttura della cartella contenente le configurazioni.
~\dirtree{%
    .1 /                (/etc/mosquitto/).
    .2 mosquitto.conf   (configurazione di default, da non modificare).
    .2 conf.d           (cartella per la configurazione .conf).
    .2 certs            (cartella per i certificati del server).
    .2 ca\_certificates  (cartella per i certificati della C.A.).
}

\subsection{Configurazione iniziale}
La configurazione del broker al momento dell'installazione è data dal file mosquitto.conf.
\newline
Se la cartella conf.d non contiene un file .conf viene caricata una configurazione di default
propria del broker. È impostata come segue:
\begin{lstlisting}[language=text]
listener 1883  #abilita il broker in ascolto alla porta 1883
allow_anonymous true #abilita l'accesso al broker a chiunque
\end{lstlisting}
Va sottolineato che con questa configurazione l'accesso al broker è abilitato a chiunque
e non ci sono restrizioni, perciò deve essere usato solo come ambiente di sviluppo.

\subsection{Configurazione parametri}
\label{subsec:broker-esempi-configurazione}
Può capitare che l'installazione non crei la struttura dei file e delle cartelle.
Si deve procedere a creare nella cartella \textbf{/etc/mosquitto} almeno il file
mosquitto.conf (si veda~\ref{sec:/mosquitto.conf}) e la cartella conf.d.\newline
Alcuni esempi di configurazione sono presenti in:
\begin{lstlisting}[language=textnonum]
    /usr/share/doc/mosquitto/examples
\end{lstlisting}
Per configurare il server in base alle proprie necessità, si deve creare un file con estensione .conf
all'interno della cartella conf.d. Questa sarà la configurazione che verrà caricata quando si eseguirà
il riavvio di Mosquitto.\newline
Di seguito vengono riportati alcuni parametri rilevanti che sono stati usati.
Si può fare riferimento alla documentazione ufficiale per la lista completa \cite{mosquitto-par-conf}.
\begin{itemize}
    \item \textbf{per\_listener\_settings [ true|false ]} :\label{table:broker-per-listener}\newline
    Abilita la gestione di autenticazione in modo differente per ogni listener.
    In questo modo si può avere, ad esempio, localhost senza password, mentre per l'accesso esterno serve
    l'autenticazione. Di default è False.
    \item \textbf{listener (port-number) [ip address/host name/unix socket path]} :\newline
    Dice a Mosquitto su quale porta mettersi in ascolto per le connessioni in entrata.
    Si deve specificare il numero della porta (le porte consigliate sono 1883 senza SSL e 8883 con SSL).
    Inoltre si deve associare il listener a un indirizzo ip o ad un host o ad un socket.
    Si possono definire più listener, uno per ogni porta, ed è possibile avere impostazioni diverse per ogni listener.
    Una volta definito un listener, le successive configurazioni sono associate a quel listener.
    \item \textbf{allow\_anonymous [ true|false ]} :\newline
    Determina se i client senza username sono abilitati alla connessione.
    Se è false si devono aggiungere configurazioni per il sistema di autenticazione.
    Viene associato ad un listener.
    Di default è False.
    \item \textbf{password\_file (file-path)} :\newline
    Imposta il percorso del password file.
    Il file contiene la coppia utente:password per ogni riga, con la password salvata come hash.
    È un primo metodo di autenticazione e si associa ad un listener.
    Per ulteriori dettagli si veda \ref{subsec:broker-password-file}.
    \item \textbf{certfile (file-path)} :\newline
    Imposta il percorso del certificato PEM del server.
    Viene usato insieme al parametro keyfile per abilitare la crittografia su TLS.
\end{itemize}
Nell'appendice~\ref{sec:/conf.d/localhost-tls.config} viene riporta la configurazione applicata al broker Mosquitto.
Sono presenti altri parametri che vengono spiegati successivamente.


\subsection{Metodi di autenticazione}
\label{sec:broker-autenticazione}
Come descritto nella guida ufficiale \cite{mosquitto-auth}, esistono 3 tipi di autenticazione:
\begin{itemize}
    \item unauthorized/anonymous access: nessun controllo.
    \item password files: gestione basilare.
    \item authentication plugins: controllo più avanzato.
\end{itemize}
Inoltre si possono avere differenti metodi di autenticazione per ogni listener, configurando più parametri listener,
\ref{table:broker-per-listener}.\newline
Nel progetto è stata usata l'autenticazione tramite \textit{password files} dato l'esiguo numero di account richiesti.


\subsubsection{Password file}
\label{subsec:broker-password-file}
Questo è un metodo semplice per gestire gli utenti in un singolo file,
tuttavia non è scalabile perché non permette la gestione puntuale di ogni client.
Inoltre richiede l'esecuzione di comandi dal terminale,
quindi è applicabile per casi non complessi come questo progetto.\newline
Questo metodo richiede le seguenti caratteristiche:
\begin{itemize}
    \item l'utente deve avere un nome univoco.
    \item la password deve essere immessa da terminale o in un file in chiaro.
    Una volta inserita viene salvato il suo hash.
    \item nel file di configurazione si devono applicare queste modifiche:
    \begin{itemize}
        \item \textbf{allow\_anonymous false} \newline
        Per disabilitare l'accesso anonimo e usare solo l'autenticazione utente e password.
        \item \textbf{password\_file /etc/mosquitto/nomefile} \newline
        Per dire al broker il percorso del file contente gli utenti e le password validi per quel listener.
    \end{itemize}
\end{itemize}
Per la creazione del file e degli utenti si possono eseguire i seguenti comandi:
\begin{itemize}
    \item \textbf{mosquitto\_passwd -c nomefile nomeutente} \newline
    Permette di creare nel percorso corrente un file con nome \textit{nomefile} e un utente con nome \textit{nomeutente}.
    Viene richiesto l'inserimento della relativa password. Questo metodo può essere usato per aggiungere un utente senza
    sovrascrivere il file se esiste già.
    \item \textbf{mosquitto\_passwd -b nomefile nomeutente password} \newline
    Permette l'aggiunta di un utente e della relativa password al file indicato.
    \item \textbf{mosquitto\_passwd -D nomefile nomeutente} \newline
    Permette l'eliminazione di un utente e della relativa password dal file indicato.
    \item \textbf{mosquitto\_passwd -U nomefile} \newline
    Dato un file in qualsiasi formato contenete utente e password in chiaro, uno per ogni riga,
    esegue l'hash delle password. È un metodo più veloce per aggiungere utenti.
\end{itemize}
Terminate le modifiche si deve ricaricare la configurazione del broker.
Per ricaricare la configurazione senza riavviare il servizio, si devono avere i diritti root:
\begin{lstlisting}[language=bash]
sudo pkill -HUP -x mosquitto
\end{lstlisting}
Per l'elenco completo dei comandi si guardi la documentazione di Mosquitto \cite{mosquitto-password}.
La guida che spiega come implementare la connessione al broker da parte di un client con autenticazione
è riportata nella sitografia finale \cite{mosquitto-password-1}.


\subsection{Certificati SSL/TLS}
\label{sec:broker-ssl-tls}
Il broker Mosquitto supporta l'autenticazione e la connessione criptata tramite SSL.\newline
È possibile generare dei certificati self signed, come spiegato nella documentazione \cite{mosquitto-ssl}.
Tuttavia si possono incontrare programmi o script che non gradiscono questo tipo di certificati, quindi
si deve optare per dei certificati di una vera CA, Certificate Authority.\newline
Si è optato per la configurazione dei certificati tramite la CA LetsEncrypt. Di seguito vengono spiegati
i vari passaggi di configurazione di \textit{certbot} per LetsEncrypt, per poi soffermarsi sul funzionamento
con il broker e i relativi problemi.\newline
Vengono riportate man mano varie fonti, dato che non è stato possibile trovare un'unica documentazione per risolvere
i problemi \cite{mosquitto-ssl-1} \cite{mosquitto-ssl-2} \cite{mosquitto-ssl-3}.

\subsubsection{Certbot}
Certbot è un tool che permette di gestire i certificati, le richieste e i rinnovi.
Per la creazione del certificato si deve avere l'accesso root alla macchina.
Inoltre, è necessario avere l'accesso alle impostazioni DNS del dominio associato alla macchina,
oppure, in alternativa, un web server sulla macchina in cui si installa Mosquitto.\newline
Si consiglia di seguire la documentazione per la corretta configurazione ad ogni passaggio \cite{mosquitto-ssl-1}.\newline
La procedura prevede:
\begin{enumerate}
    \item \textbf{Installazione Certbot}\newline
    Si esegue il comando:
    \begin{lstlisting}[language=bash]
    sudo apt-get -y install certbot
    \end{lstlisting}
    \item \textbf{Creazione di un certificato tramite DNS}\newline
    Il certificato che si crea è di tipo wildcard, quindi se il dominio è mydomain.it è valido anche per i sottodomini,
    così da poterlo usare per altri servizi.
    Il dominio DNS deve essere già associato alla macchina.
    Si esegue il comando:
    \begin{lstlisting}[language=bash]
    certbot certonly --manual --preferred-challenge dns -d *.mydomain.it
    \end{lstlisting}
    Si seguono le istruzioni a schermo, che prevedono l'aggiunta di un record DNS al dominio per validarlo.
    Al termine si ha generato dei certificati validi.
    \item \textbf{Rinnovo dei certificati}\newline
    Il rinnovo si può eseguire:
    \begin{itemize}
        \item \textbf{Manualmente}\newline
        Si esegue il comando:
        \begin{lstlisting}[language=bash]
            sudo certbot renew
        \end{lstlisting}
        \item \textbf{Automaticamente}\newline
        Si deve modificare il crontab per eseguire il comando di auto rinnovo periodicamente \cite{mosquitto-ssl-4}.
        Si esegue:
        \begin{lstlisting}[language=bash]
            sudo crontab-e
        \end{lstlisting}
        Poi si aggiungono alla fine del file le seguenti righe:
        \begin{lstlisting}[language=text]
            # Auto-renew let's encrypt SSL certificates
            0     *     *     *     *      sudo certbot renew
        \end{lstlisting}
    \end{itemize}
    \item \textbf{Accesso ai certificati}\newline
    Per accedere ai certificati si usano i symbolic link presenti nella cartella:
    \begin{lstlisting}[language=textnonum]
    /etc/letsencrypt/live/mydomain.it/
    \end{lstlisting}
    Non si accede direttamente ai certificati perché questi hanno una scadenza, di solito di alcuni mesi.
    Questi link sono mantenuti da Certbot e puntano all'ultima versione dei certificati.
\end{enumerate}

\subsubsection{Funzionamento con Mosquitto}
Mosquitto funziona sulla macchina con un utente che non ha i diritti root,
quindi non ha accesso alle cartelle dei certificati.
Tuttavia, seguendo le indicazioni fornite su Github dalla stessa Eclipse Mosquitto,
si riesce a copiare i certificati permettendo l'accesso a Mosquitto \cite{mosquitto-ssl-2}.
La procedura prevede:
\begin{enumerate}
    \item \textbf{Copia automatica dei certificati}\newline
    Lo script presente nell'appendice~\ref{sec:script-rinnovo-certificati} permette la copia e il cambio proprietario
    ad ogni rinnovo dei certificati.
    Lo script deve essere modificato nella riga 16, per specificare il dominio dei certificati interessati,
    e nella riga 18, per specificare la cartella di Mosquitto in cui si copieranno i certificati.
    Successivamente si copia lo script nella cartella (servono i diritti root):
    \begin{lstlisting}[language=textnonum]
    /etc/letsencrypt/renewal-hooks/deploy/
    \end{lstlisting}
    Si rende eseguibile lo script:
    \begin{lstlisting}[language=bash]
        sudo chmod +x /etc/letsencrypt/renewal-hooks/deploy/mosquitto-copy.sh
    \end{lstlisting}
    Con questa procedura Mosquitto ha ad ogni rinnovo dei certificati l'ultima versione.
    \item \textbf{Copia manuale dei certificati}\newline
    Probabilmente la prima volta si devono copiare manualmente i certificati.
    Si devono eseguire i seguenti comandi che fanno gli stessi passaggi dello script:
    \begin{lstlisting}[language=bash]
        # copia dei file
        sudo cp  /etc/letsencrypt/live/mydomain.it/fullchain.pem /etc/mosquitto/certs/fullchain.pem
        sudo cp  /etc/letsencrypt/live/mydomain.it/privkey.pem /etc/mosquitto/certs/privkey.pem
        # imposta l'owner a Mosquitto
        sudo chown mosquitto: /etc/mosquitto/certs/fullchain.pem /etc/mosquitto/certs/privkey.pem
        # imposta i diritti solo per Mosquitto
        sudo chmod 0600 /etc/mosquitto/certs/fullchain.pem /etc/mosquitto/certs/privkey.pem
        # dice a Mosquitto di ricaricare certificati e configurazione
        sudo pkill -HUP -x mosquitto
    \end{lstlisting}
    \item \textbf{Modifica configurazione Mosquitto}\newline
    Nella configurazione .conf di mosquitto si devono modificare 2 parametri impostando il path dei certificati:
    \begin{lstlisting}[language=bash]
        certfile /etc/mosquitto/certs/fullchain.pem
        keyfile /etc/mosquitto/certs/privkey.pem
    \end{lstlisting}
\end{enumerate}
Terminate queste modifiche Mosquitto può ricevere connessioni con TSL sul listener configurato.
I client che si connettono devono abilitare la connessione con TSL e opzionalmente la validazione del certificato.

\section{Disinstallazione}
\label{sub:broker-disinstallazione-del-broker}

\subsection{Eliminazione della configurazione}
\label{subsec:broker-eliminazione-configurazione}
Prima di procedere con la disinstallazione si consiglia di fare il backup delle configurazioni
e dei certificati presenti nelle rispettive cartelle.
Successivamente si consiglia di eliminare la cartella \textbf{/etc/mosquitto} per evitare futuri problemi.
Nel caso in cui siano stati configurati listener con TLS,
si deve eliminare lo script di copia automatica dei certificati.

\subsection{Rimozione del broker}
\label{subsec:broker-disinstallazione}
Per rimuovere la repository e l'installazione di Mosquitto si eseguono i seguenti comandi:
\begin{lstlisting}[language=bash]
sudo apt-add-repository --remove ppa:mosquitto-dev/mosquitto-ppa
sudo apt-get update
sudo apt-get remove mosquitto
sudo apt-get autoremove
\end{lstlisting}
